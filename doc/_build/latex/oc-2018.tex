%% Generated by Sphinx.
\def\sphinxdocclass{report}
\documentclass[letterpaper,10pt,french]{sphinxmanual}
\ifdefined\pdfpxdimen
   \let\sphinxpxdimen\pdfpxdimen\else\newdimen\sphinxpxdimen
\fi \sphinxpxdimen=.75bp\relax

\PassOptionsToPackage{warn}{textcomp}
\usepackage[utf8]{inputenc}
\ifdefined\DeclareUnicodeCharacter
 \ifdefined\DeclareUnicodeCharacterAsOptional
  \DeclareUnicodeCharacter{"00A0}{\nobreakspace}
  \DeclareUnicodeCharacter{"2500}{\sphinxunichar{2500}}
  \DeclareUnicodeCharacter{"2502}{\sphinxunichar{2502}}
  \DeclareUnicodeCharacter{"2514}{\sphinxunichar{2514}}
  \DeclareUnicodeCharacter{"251C}{\sphinxunichar{251C}}
  \DeclareUnicodeCharacter{"2572}{\textbackslash}
 \else
  \DeclareUnicodeCharacter{00A0}{\nobreakspace}
  \DeclareUnicodeCharacter{2500}{\sphinxunichar{2500}}
  \DeclareUnicodeCharacter{2502}{\sphinxunichar{2502}}
  \DeclareUnicodeCharacter{2514}{\sphinxunichar{2514}}
  \DeclareUnicodeCharacter{251C}{\sphinxunichar{251C}}
  \DeclareUnicodeCharacter{2572}{\textbackslash}
 \fi
\fi
\usepackage{cmap}
\usepackage[T1]{fontenc}
\usepackage{amsmath,amssymb,amstext}
\usepackage{babel}
\usepackage{times}
\usepackage[Sonny]{fncychap}
\ChNameVar{\Large\normalfont\sffamily}
\ChTitleVar{\Large\normalfont\sffamily}
\usepackage{sphinx}

\usepackage{geometry}

% Include hyperref last.
\usepackage{hyperref}
% Fix anchor placement for figures with captions.
\usepackage{hypcap}% it must be loaded after hyperref.
% Set up styles of URL: it should be placed after hyperref.
\urlstyle{same}
\addto\captionsfrench{\renewcommand{\contentsname}{Contents:}}

\addto\captionsfrench{\renewcommand{\figurename}{Fig.}}
\addto\captionsfrench{\renewcommand{\tablename}{Tableau}}
\addto\captionsfrench{\renewcommand{\literalblockname}{Code source}}

\addto\captionsfrench{\renewcommand{\literalblockcontinuedname}{suite de la page précédente}}
\addto\captionsfrench{\renewcommand{\literalblockcontinuesname}{suite sur la page suivante}}

\addto\extrasfrench{\def\pageautorefname{page}}

\setcounter{tocdepth}{1}



\title{oc-2018 Documentation}
\date{août 01, 2018}
\release{0.1}
\author{Raphael Holzer}
\newcommand{\sphinxlogo}{\vbox{}}
\renewcommand{\releasename}{Version}
\makeindex

\begin{document}

\maketitle
\sphinxtableofcontents
\phantomsection\label{\detokenize{index::doc}}



\chapter{Option complémentaire en informatique}
\label{\detokenize{01-intro:welcome-to-oc-2018-s-documentation}}\label{\detokenize{01-intro::doc}}\label{\detokenize{01-intro:option-complementaire-en-informatique}}
Once Sphinx is {\hyperref[\detokenize{glossary::doc}]{\sphinxcrossref{\DUrole{doc}{installed}}}}, you can proceed with
setting up your first Sphinx project. To ease the process of getting started,
Sphinx provides a tool, \sphinxstyleliteralstrong{\sphinxupquote{sphinx-quickstart}}, which will generate a
documentation source directory and populate it with some defaults. We’re going
to use the \sphinxstyleliteralstrong{\sphinxupquote{sphinx-quickstart}} tool here, though it’s use by no means
necessary.


\section{Préparation de l’infrastructure}
\label{\detokenize{01-intro:preparation-de-l-infrastructure}}
The root directory of a Sphinx collection of {\hyperref[\detokenize{glossary:term-restructuredtext}]{\sphinxtermref{\DUrole{xref,std,std-term}{reStructuredText}}}} document
sources is called the {\hyperref[\detokenize{glossary:term-source-directory}]{\sphinxtermref{\DUrole{xref,std,std-term}{source directory}}}}.  This directory also contains
the Sphinx configuration file \sphinxcode{\sphinxupquote{conf.py}}, where you can configure all
aspects of how Sphinx reads your sources and builds your documentation.

\fvset{hllines={, ,}}%
\begin{sphinxVerbatim}[commandchars=\\\{\}]
\PYGZdl{} pip install jupyter
\PYGZdl{} jupyter labextension list
\end{sphinxVerbatim}

Sphinx comes with a script called \sphinxstyleliteralstrong{\sphinxupquote{sphinx-quickstart}} that sets up a
source directory and creates a default \sphinxcode{\sphinxupquote{conf.py}} with the most useful
configuration values from a few questions it asks you. To use this, run:

\fvset{hllines={, ,}}%
\begin{sphinxVerbatim}[commandchars=\\\{\}]
\PYGZdl{} sphinx\PYGZhy{}quickstart
\end{sphinxVerbatim}
\index{enumerate() (fonction de base)}

\begin{fulllineitems}
\phantomsection\label{\detokenize{01-intro:enumerate}}\pysiglinewithargsret{\sphinxbfcode{\sphinxupquote{enumerate}}}{\emph{sequence}\sphinxoptional{, \emph{start=0}}}{}~\begin{quote}

Return an iterator that yields tuples of an index and an item of the
\sphinxstyleemphasis{sequence}. (And so on.)
\end{quote}

The {\hyperref[\detokenize{01-intro:enumerate}]{\sphinxcrossref{\sphinxcode{\sphinxupquote{enumerate()}}}}} function can be used for …

\end{fulllineitems}



\chapter{Glossary}
\label{\detokenize{glossary:id1}}\label{\detokenize{glossary::doc}}\label{\detokenize{glossary:glossary}}\begin{description}
\item[{builder\index{builder|textbf}}] \leavevmode\phantomsection\label{\detokenize{glossary:term-builder}}
A class (inheriting from \sphinxcode{\sphinxupquote{Builder}}) that takes
parsed documents and performs an action on them.  Normally, builders
translate the documents to an output format, but it is also possible to
use the builder builders that e.g. check for broken links in the
documentation, or build coverage information.

See {\hyperref[\detokenize{glossary::doc}]{\sphinxcrossref{\DUrole{doc}{Glossary}}}} for an overview over Sphinx’s built-in
builders.

\item[{configuration directory\index{configuration directory|textbf}}] \leavevmode\phantomsection\label{\detokenize{glossary:term-configuration-directory}}
The directory containing \sphinxcode{\sphinxupquote{conf.py}}.  By default, this is the same as
the {\hyperref[\detokenize{glossary:term-source-directory}]{\sphinxtermref{\DUrole{xref,std,std-term}{source directory}}}}, but can be set differently with the \sphinxstylestrong{-c}
command-line option.

\item[{source directory\index{source directory|textbf}}] \leavevmode\phantomsection\label{\detokenize{glossary:term-source-directory}}
The directory which, including its subdirectories, contains all source
files for one Sphinx project.

\item[{reStructuredText\index{reStructuredText|textbf}}] \leavevmode\phantomsection\label{\detokenize{glossary:term-restructuredtext}}
An easy-to-read, what-you-see-is-what-you-get plaintext markup syntax and
parser system.

\end{description}


\chapter{Indices and tables}
\label{\detokenize{index:indices-and-tables}}\begin{itemize}
\item {} 
\DUrole{xref,std,std-ref}{genindex}

\item {} 
\DUrole{xref,std,std-ref}{modindex}

\item {} 
\DUrole{xref,std,std-ref}{search}

\end{itemize}



\renewcommand{\indexname}{Index}
\printindex
\end{document}