%% Generated by Sphinx.
\def\sphinxdocclass{report}
\documentclass[letterpaper,10pt,french]{sphinxmanual}
\ifdefined\pdfpxdimen
   \let\sphinxpxdimen\pdfpxdimen\else\newdimen\sphinxpxdimen
\fi \sphinxpxdimen=.75bp\relax

\PassOptionsToPackage{warn}{textcomp}
\usepackage[utf8]{inputenc}
\ifdefined\DeclareUnicodeCharacter
 \ifdefined\DeclareUnicodeCharacterAsOptional
  \DeclareUnicodeCharacter{"00A0}{\nobreakspace}
  \DeclareUnicodeCharacter{"2500}{\sphinxunichar{2500}}
  \DeclareUnicodeCharacter{"2502}{\sphinxunichar{2502}}
  \DeclareUnicodeCharacter{"2514}{\sphinxunichar{2514}}
  \DeclareUnicodeCharacter{"251C}{\sphinxunichar{251C}}
  \DeclareUnicodeCharacter{"2572}{\textbackslash}
 \else
  \DeclareUnicodeCharacter{00A0}{\nobreakspace}
  \DeclareUnicodeCharacter{2500}{\sphinxunichar{2500}}
  \DeclareUnicodeCharacter{2502}{\sphinxunichar{2502}}
  \DeclareUnicodeCharacter{2514}{\sphinxunichar{2514}}
  \DeclareUnicodeCharacter{251C}{\sphinxunichar{251C}}
  \DeclareUnicodeCharacter{2572}{\textbackslash}
 \fi
\fi
\usepackage{cmap}
\usepackage[T1]{fontenc}
\usepackage{amsmath,amssymb,amstext}
\usepackage{babel}
\usepackage{times}
\usepackage[Sonny]{fncychap}
\ChNameVar{\Large\normalfont\sffamily}
\ChTitleVar{\Large\normalfont\sffamily}
\usepackage{sphinx}

\usepackage{geometry}

% Include hyperref last.
\usepackage{hyperref}
% Fix anchor placement for figures with captions.
\usepackage{hypcap}% it must be loaded after hyperref.
% Set up styles of URL: it should be placed after hyperref.
\urlstyle{same}
\addto\captionsfrench{\renewcommand{\contentsname}{Contents:}}

\addto\captionsfrench{\renewcommand{\figurename}{Fig.}}
\addto\captionsfrench{\renewcommand{\tablename}{Tableau}}
\addto\captionsfrench{\renewcommand{\literalblockname}{Code source}}

\addto\captionsfrench{\renewcommand{\literalblockcontinuedname}{suite de la page précédente}}
\addto\captionsfrench{\renewcommand{\literalblockcontinuesname}{suite sur la page suivante}}

\addto\extrasfrench{\def\pageautorefname{page}}

\setcounter{tocdepth}{1}



\title{oc-2018 Documentation}
\date{août 01, 2018}
\release{0.1}
\author{Raphael Holzer}
\newcommand{\sphinxlogo}{\vbox{}}
\renewcommand{\releasename}{Version}
\makeindex

\begin{document}

\maketitle
\sphinxtableofcontents
\phantomsection\label{\detokenize{index::doc}}



\chapter{Option complémentaire en informatique}
\label{\detokenize{01-intro:welcome-to-oc-2018-s-documentation}}\label{\detokenize{01-intro:option-complementaire-en-informatique}}\label{\detokenize{01-intro::doc}}
Once Sphinx is \DUrole{xref,std,std-doc}{installed}, you can proceed with
setting up your first Sphinx project. To ease the process of getting started,
Sphinx provides a tool, \sphinxstyleliteralstrong{\sphinxupquote{sphinx-quickstart}}, which will generate a
documentation source directory and populate it with some defaults. We’re going
to use the \sphinxstyleliteralstrong{\sphinxupquote{sphinx-quickstart}} tool here, though it’s use by no means
necessary.


\section{Préparation de l’infrastructure}
\label{\detokenize{01-intro:preparation-de-l-infrastructure}}
The root directory of a Sphinx collection of \DUrole{xref,std,std-term}{reStructuredText} document
sources is called the \DUrole{xref,std,std-term}{source directory}.  This directory also contains
the Sphinx configuration file \sphinxcode{\sphinxupquote{conf.py}}, where you can configure all
aspects of how Sphinx reads your sources and builds your documentation.  {\color{red}\bfseries{}{[}\#{]}\_}

Sphinx comes with a script called \sphinxstyleliteralstrong{\sphinxupquote{sphinx-quickstart}} that sets up a
source directory and creates a default \sphinxcode{\sphinxupquote{conf.py}} with the most useful
configuration values from a few questions it asks you. To use this, run:

\fvset{hllines={, ,}}%
\begin{sphinxVerbatim}[commandchars=\\\{\}]
\PYGZdl{} sphinx\PYGZhy{}quickstart
\end{sphinxVerbatim}


\chapter{Indices and tables}
\label{\detokenize{index:indices-and-tables}}\begin{itemize}
\item {} 
\DUrole{xref,std,std-ref}{genindex}

\item {} 
\DUrole{xref,std,std-ref}{modindex}

\item {} 
\DUrole{xref,std,std-ref}{search}

\end{itemize}



\renewcommand{\indexname}{Index}
\printindex
\end{document}